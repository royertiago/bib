\documentclass{article}

\usepackage{complexity}

\begin{document}
\title{Notes about \cite{Cook1973}}
\author{Tiago Royer}
\maketitle

\textbf{Page 2:} The notation $L_N(T(n))$ that Cook uses
is equivalent to $\NTIME(T(n))$.

I think that Cook's \emph{real time countable}
is what textbooks (like \cite{HopcroftUllman1979} and~\cite{AroraBarak2009})
call ``time-constructible''.
He requires the function to output a string of size $f(n)$
(in the output tape)
using only $f(n)$ time;
so, the machine \emph{must} output one symbol per step,
thus requiring the machine to function for exactly $f(n)$ steps.
This makes his definition essentially the same as
Hopcroft's and Ullman's~\cite[p.~299]{HopcroftUllman1979}.

\bibliographystyle{plain}
\bibliography{bibliography}

\end{document}
